\section{Introducci\'on}
    
    Los protocolos confiables de transporte permiten al usuario abstraerse 
    de los problemas de transmisi\'on de la red, por ej, el protocolo TCP 
    nos permite enviar mensajes correctamente a trav\'es de medios que pueden
    perder, desordenar o incluso modificar paquetes. 
    
    El protocolo PTC es una implementaci\'on de la catedra, el mismo est\'a 
    basado en TCP, por lo tanto, intenta asegurar que los paquetes lleguen de un
    punto a otro utilizando diversos mecanismos como control de flujo, realizar
    la conexi\'on en etapas, mantener una ventana deslizante para enviar 
    y recibir paquetes, etc. 
        
    En este trabajo estudiaremos como se comporta PTC, poniendo especial 
    atenci\'on a la retransmisi\'on de paquetes. Para ello 
    primero emularemos escenarios donde existan delay y perdida
    de paquetes entre dos nodos comunicados. Luego, iremos modificando
    par\'ametros de la implementaci\'on para ver como ello afecta el reenv\'io 
    de paquetes. 

