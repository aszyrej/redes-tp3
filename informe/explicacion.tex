    Se agregaron al sistema variables para manejar la probabilidad de que 
    un paquete se pierda en el env\'io, el delay antes de contestar con 
    un ACK, el valor de $\alpha$ y $\beta$.

\subsubsection{rto}
    Se modific\'o para que los valores de $\alpha$ y $\beta$ fueran 
    variables seteables al momento de inicializar el \textit{RTOEstimator}.
    
\subsubsection{protocol}
    Se modific\'o la incializaci\'on, ahora recibe los par\'ametros 
    $\alpha$, $\beta$, \textit{perdida} y \textit{delay}.
    
    $\alpha$, $\beta$ se utilizan para inicializar el \textit{RTOEstimator}.
    
    Se modific\'o el m\'etodo \textit{send_and_queue}, ahora, env\'ia 
    el paquete con una probabilidad de $1-perdida$.
    
    Se crearon los m\'etodos \textit{alumnos_get_delay} y 
    \textit{alumnos_change_delay}, los mismos devuelven y cambian
    respectivamente la variable \textit{delay}.
    
    Se cre\'o el m\'etodo \textit{alumnos_get_rto} que devuelve el RTO
    estimado hasta el momento.
        
\subsubsection{handler}
    Se modific\'o el m\'etodo \textit{send_ack}, ahora, espera 
    \textit{delay ticks} de reloj antes de enviar el paquete.
    
\subsubsection{ptc_socket}
    Se modific\'o el wrapper, ahora se debe inicializar con valores para 
    $\alpha$, $\beta$, \textit{perdida} y \textit{delay}, los mismos son
    utilizados al momento de crear el objeto \textit{PTCProtocol} del 
    paquete \textit{protocol}.
    
    Se cre\'o el m\'etodo \textit{alumnos_change_delay} que recibe un
    entero y cambia el delay de la conexi\'on mediante el m\'etodo nuevo
    puesto en \textit{protocol}.
    
    Se modific\'o el m\'etodo \textit{close} para que imprima el RTO 
    estimado durante la conexi\'on.
