\section{Desarrollo}

    En primera instancia se implementaron dos scripts en python que 
    actuan uno como cliente y otro como servidor.
    Luego, se modific\'o la implementaci\'on de PTC para poder simular
    delay y perdida de paquetes, ademas de transformar en par\'ametros
    a los valores $\alpha$ y $\beta$ para estudiar como se modifica el
    \rto{} en base a los mismos.  

  \subsection{Modificaciones a la Implementaci\'on}
    Se agregaron al sistema variables para manejar la probabilidad de que 
    un paquete se pierda en el env\'io, el delay antes de contestar con 
    un ACK, el valor de $\alpha$ y $\beta$.

    \subsubsection{rto}
    Se modific\'o para que los valores de $\alpha$ y $\beta$ fueran 
    variables seteables al momento de inicializar el \textit{RTOEstimator}.
    
    \subsubsection{protocol}
    Se modific\'o la incializaci\'on, ahora recibe los par\'ametros 
    $\alpha$, $\beta$, \textit{perdida} y \textit{delay}.
    
    $\alpha$, $\beta$ se utilizan para inicializar el \textit{RTOEstimator}.
    
    Se modific\'o el m\'etodo \textit{send\_and\_queue}, ahora, espera 
    \textit{delay} ticks antes de enviar el paquete con una probabilidad
    de $1-perdida$.

    Se modific\'o el m\'etodo \textit{free} para que imprima el \rto{} 
    estimado durante la conexi\'on.        
    
    \subsubsection{ptc\_socket}
    Se modific\'o el wrapper, ahora se debe inicializar con valores para 
    $\alpha$, $\beta$, \textit{perdida} y \textit{delay}, los mismos son
    utilizados al momento de crear el objeto \textit{PTCProtocol} del 
    paquete \textit{protocol}.
    
    Se cre\'o el m\'etodo \textit{alumnos\_change\_delay} que recibe un
    entero y cambia el delay de la conexi\'on mediante el m\'etodo nuevo
    puesto en \textit{protocol}.
    
  \subsection{Experimentaci\'on}  
    Para estudiar el comportamiento de PTC realizaremos los siguientes
    experimentos:
    
    \textb{Convergencia de RTO:} Queremos ver si el valor de \rto{} se
    estabiliza luego de enviar un gran n\'umero de paquetes.
  
    \textb{Valor de RTO variando \beta}
    
    \textb{Valor de RTO variando \alpha}
    
    \textb{Valor de RTO cuando la red se congestiona de golpe:} En este
    caso queremos ver que sucede cuando la red esta operando normalmente
    y de repente subimos mucho el delay.
