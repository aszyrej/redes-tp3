\subsection{Resultados}

    \subsubsection{Valor de RTO variando $\alpha$ y $\beta$}
    
        Primero estudiamos como var\'ia el \rto{} cuando se utilizan los
        valores extremos de $\alpha$ y $\beta$. Se utiliz\'o un valor 
        de delay = \textit{6 ticks} sin probabilidad de p\'erdida.

        Puede verse en la \textit{Figura 1}, que tomando $\alpha=0$, 
        $\beta=0$ los cálculos ignoran por completo los nuevos 
        valores de RTT y consideran únicamente los valores de SRTT
        anteriores. En este caso, el valor constante inicial.

        La \textit{Figura 2} muestra como al tomar ambas variables el valor
        1, los cálculos ignoran los valores anteriores (tanto de variación
        de RTT (RTTVAR) como de SRTT), y modifican los resultados
        únicamente en función de la última muestra de RTT obtenida. El
        resultado es un gráfico en donde el RTO coincide exactamente con
        el SRTT, y ambos responden excesivamente a cualquier fluctuación
        que se produzca en el valor del RTT.
    
        Por otra parte, la \textit{Figura 3} muestra lo que sucede al 
        ponderar de igual manera los valores anteriores con los nuevos, al
        tomar ambas variables con valor 0.5. Esto debería generar valores
        cercanos a los de las muestras anteriores, pero que sean capaces
        de responder a cambios significativos y permanentes del valor de
        RTT. En otras palabras, que alcancen un equilibrio entre ambas
        propuestas. Sin embargo, el comportamiento no fue el esperado. 
        
        El resultado es una serie de cálculos demasiado sensibles a las
        fluctuaciones temporales que responden excesivamente a estos
        cambios. Por ello hicimos una \'ultima prueba (\textit{Figura 4})
        donde fijamos el valor de $\alpha$ en 0.15 y $\beta$ en 0.20 
        (cercanos al propuesto por el RFC 6298).
        Las conexiones en las distintas redes por lo general permanecen
        estables la mayor parte del tiempo, por lo que los cambios bruscos
        de \rtt{} son poco probables. Por otro lado el valor de $\beta$ 
        permite agregar un poco de sensibilidad a los cambios en el valor
        de los \textit{rtts}.

        \begin{figure}[H]
	    \center
	    \begin{subfigure}{0.48\textwidth}
		    \includegraphics[width=1.0\textwidth]{imagenes/test_0_0.jpg}
		    \caption{Figura 1}
	    \end{subfigure}
	    ~
	    \begin{subfigure}{0.48\textwidth}
		    \includegraphics[width=1.0\textwidth]{imagenes/test_1_1.jpg}
		    \caption{Figura 2}
	    \end{subfigure}
	    \end{figure}    

        \begin{figure}[H]
	    \center
	    \begin{subfigure}{0.48\textwidth}
		    \includegraphics[width=1.0\textwidth]{imagenes/test_5_5.jpg}
		    \caption{Figura 3}
	    \end{subfigure}
	    ~
	    \begin{subfigure}{0.48\textwidth}
		    \includegraphics[width=1.0\textwidth]{imagenes/test_a_b.jpg}
		    \caption{Figura 4}
	    \end{subfigure}
	    \end{figure}    
	       
        Finalmente intentamos medir muchos valores de \rto{} para 
        diversas convinaciones de $\alpha$ y $\beta$. 
        Los resultados de medir el \rto{} para un valor fijo de 
        \textit{25 ticks} y probabilidad de perdida cero luego de 
        enviar 200 paquetes se pueden ver en la \emph{Figura 4}.
        
    \subsubsection{Perdida de paquetes variando $\alpha$ y $\beta$}
        Para estudiar la perdida de paquetes se fij\'o el valor de 
        de \rto{} en \textit{25 ticks} y no se simularon p\'erdidas.
        Sin embargo en varios casos hubo retransmisiones, la 
        \textit{Figura 5} muestra el resultado para el env\'io de 150
        paquetes mientras que la \textit{Figura 6} lo muestra para 300.
        
    \begin{figure}[H]
	    \center
	    \begin{subfigure}{0.32\textwidth}
		    \includegraphics[width=1.0\textwidth]{imagenes/rto_vs_alphaBeta.pdf}
		    \caption{Figura 4}
	    \end{subfigure}
	    ~
	    \begin{subfigure}{0.32\textwidth}
		    \includegraphics[width=1.0\textwidth]{imagenes/retransmisiones_150.pdf}
		    \caption{Figura 5}
	    \end{subfigure}
	    ~
	    \begin{subfigure}{0.32\textwidth}
		    \includegraphics[width=1.0\textwidth]{imagenes/retransmisiones_300.pdf}
		    \caption{Figura 6}
	    \end{subfigure}
	
    \end{figure}    


    \subsubsection{Valor de RTO cuando la red se congestiona sin perdida}
        Para este configuramos el cliente para que env\'ie 300 paquetes al
        servidor, pero luego de enviarse 150 paquetes, el delay de la red
        se duplicara o cuadruplicara. 
        
        El \emph{Escenario 1} muestra el resultado en una red con
        par\'aemtros: $\alpha={{1}\over{2}}$, $\beta={{1}\over{4}}$
        cuando no se congestiona, cuando la congesti\'on causa el doble de
        delay, y cuando causa el cuadruple.
  
        \begin{figure}[H]
            \center
	        
		    \includegraphics[width=0.32\textwidth]{imagenes/rtt_vs_n_2_4.pdf}
		    \includegraphics[width=0.32\textwidth]{imagenes/congestion_50_2_4.pdf}
		    \includegraphics[width=0.32\textwidth]{imagenes/congestion_100_2_4.pdf}

            \caption{Escenario 1}
	
        \end{figure}          
  
        El \emph{Escenario 2} muestra el resultado en una red con
        par\'aemtros: $\alpha={{1}\over{8}}$, $\beta={{1}\over{2}}$ 
        cuando no se congestiona, cuando la congesti\'on causa el doble de
        delay, y cuando causa el cuadruple.

        \begin{figure}[H]
            \center
	        
		    \includegraphics[width=0.32\textwidth]{imagenes/rtt_vs_n_8_2.pdf}
		    \includegraphics[width=0.32\textwidth]{imagenes/congestion_50_8_2.pdf}
		    \includegraphics[width=0.32\textwidth]{imagenes/congestion_100_8_2.pdf}

            \caption{Escenario 1}
	
        \end{figure}          

        En ambos casos sucedi\'o que al duplicar el valor del delay el 
        mensaje n\'umero 101 debi\'o ser retransmitido. A su vez, cuando 
        se cuadruplic\'o el valor de la red, el mensaje n\'umero 101 debi\'o
        ser retransmitido dos veces. 
        Tambi\'en se puede notar que en ambos casos luego de cuadruplicarse
        el delay no se logr\'o volver a un \rto{} cercano a el \rtt{} real.


    \subsubsection{Tiempo total de transmisi\'on en entorno con p\'erdidas}
        
        Para este experimento se configur\'o un escenario en el cual el 
        cliente env\'ia 300 mensajes al servidor, pasados los 150 el delay
        se incrementa de \textit{25 ticks} a \textit{50 ticks}. A su vez,
        la probabilidad de perder un paquete es de 0.1.
        
        Para los valores de $\alpha=0$ y $\beta=0$ el tiempo fue 42,19 seg,
        para los valores propuestos por el RFC el tiempo promedio fue
        41,67 seg, para $\alpha=0.15$ y $\beta=0.2$ 42,39 seg, finalmente  
        para $\alpha=1$ y $\beta=1$ el tiempo fue 44,34 seg.
        
        Podemos ver que entre el primer resultado hay casi dos segundos 
        de diferencia, teniendo en cuenta que el tiempo para enviar un 
        mensaje entre ambos nodos es menor a 0.1 seg, podemos ver que la 
        diferencia de performance es notable.
