\section{Conclusiones}

Todos los experimentos demostraron lo susceptible que es la medici\'on de
RTO a los par\'ametros de la red, as\'i como tambi\'en a los par\'ametros
utilizados para estimarlo. 

Primero vimos que cambiar los valores de $\alpha$ y $\beta$ modifican
significativamente el valor estimado de \rto{}, al experimentar con valores
muy peque\~nos, los cálculos no se adaptan de forma realista a las
variaciones en los valores de RTT reales. Por el contrario, al tomar
valores muy altos los mismos se ven demasiado influenciados por
fluctuaciones temporales, dando como resultado valores demasiado altos o
demasiado peque\~nos en comparación con el RTT real. 

Luego vimos como los valores de $\alpha$ y $\beta$ modifican la cantidad
de retransmisiones, cabe destacar que las \textit{Figuras 5 y 6} son parte
de un experimento en el cual no se simul\'o perdida de paquetes, sin embargo
, hubo retransmisiones. Esto se debe a que cuando el \rto{} se acerca al 
\rtt{} estimado es muy sensible a cambios internos de la pc, recordemos que
los procesos de cliente y servidor est\'an funcionando dentro de un sistema
operativo hogareño. 

Por \'ultimo, vimos como el c\'alculo del \rto{} afecta a la transmisi\'on
total de una conexi\'on. Podemos ver que entre el primer resultado hay casi
dos segundos de diferencia, teniendo en cuenta que el tiempo para enviar un 
mensaje entre ambos nodos es menor a 0.1 seg, podemos ver que la diferencia
de performance es notable.

Para concluir queremos agregar que el valor que mejor funcion\'o en los 
experimentos fue el propuesto por el RFC, consiguiendo aproximarnos con $\alpha=0.15$ y $\beta=0.2$. 
