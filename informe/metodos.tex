\section{M\'etodos}
    
    El protocolo PTC utiliza un sistema de \textit{acknowledge} para
    saber si un paquete fue recibido. En caso de no recibir un \textit{ack}
    para un  determinado paquete luego de un umbral de tiempo, procede a
    retransmitirlo. Dicho umbral, denominado \textit{rto} se va estimando
    a medida que se reconocen los paquetes enviados usando las 
    siguientes funciones: 
    
        $$RTTVAR = (1-\beta)RTTBAR + \beta |SRTT-RTT|$$
        $$SRTT = (1-\alpha)SRTT + \alpha RTT$$
        $$RTO = SRTT + max( 1, K * RTTVAR) $$ 
        
    Donde $K, \alpha, \beta \in \mathbb{Q}$, $RTT$ es el valor de 
    \textit{rtt} calculado para el \'ultimo \textit{ack} recibido. 
    En la implementaci\'on utilizada se inicializan los valores de la
    siguiente manera: $K = 4, \alpha={{1}\over{8}}$,
    $\beta={{1}\over{4}}$, $SRTT={RTT_0}$, $RTTVAR={{RTT}\over{2}}$.
    Siendo $RTT_0$ el primer \textit{rtt} calculado.
    
    
