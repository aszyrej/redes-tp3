\section{M\'etodos}

    El protocolo PTC utiliza un sistema de \textit{acknowledge} para
    saber si un paquete fue recibido. En caso de no recibir un \textit{ack}
    para un  determinado paquete luego de un umbral de tiempo, procede a
    retransmitirlo. Dicho umbral, denominado \textit{rto} se va estimando
    a medida que se reconocen los paquetes enviados usando las
    siguientes funciones:

        $$RTTVAR = (1-\beta)RTTBAR + \beta |SRTT-RTT|$$
        $$SRTT = (1-\alpha)SRTT + \alpha RTT$$
        $$RTO = SRTT + max( 1, K * RTTVAR) $$

    Donde $K, \alpha, \beta \in \mathbb{Q}$, $RTT$ es el valor de
    \textit{rtt} calculado para el \'ultimo \textit{ack} recibido.
    En la implementaci\'on utilizada se inicializan los valores de la
    siguiente manera: $K = 4, \alpha={{1}\over{8}}$,
    $\beta={{1}\over{4}}$, $SRTT={RTT_0}$, $RTTVAR={{RTT}\over{2}}$.
    Siendo $RTT_0$ el primer \textit{rtt} calculado.

\subsection{Sistema}

	Tanto el cliente como el servidor est\'an simulados en el mismo sistema. La
	topolog\'ia de este sistema es similar a la de dos equipos, un cliente y un
	servidor, conectados mediante un link ethernet en la misma red o en
	diferentes redes pero con cierto nivel previsible de congesti\'on.

	Al protocolo PTC se le puede artificialmente poner un \emph{delay} ficticio,
	que simular\'ia tanto la distancia que hay entre en cliente y el servidor en
	la red simulada como el nivel de congesti\'on entre estos dos puntos.
  Por ejemplo, en los entornos reales (muy distintos al de nuestra experimentaci\'on),
  el delay incluye el tiempo en el que
  los paquetes permanecen en los buffers de cada router antes de ser forwardeados,
  el cual puede ser un factor a tener en cuenta para medir el nivel de congesti\'on
  de una red.
	Tambi\'en se tiran adrede los paquetes con una probabilidad $p$, simulando
	la perdida de paquetes que puede pasar en una red congestionada o con links
	erroneos.

	Dada la velocidad de procesamiento actual, sumado al procesamiento en
	paralelo, podemos asumir que el RTT real es el delay ficticio que generamos,
	dado que un tick simulado ronda el orden del mill\'on de tick reales.

	En lo que respecta a este TP, todas las conexiones son conexiones que usan
	el protocolo PTC.
